\documentclass[a4paper]{article}


\usepackage[utf8]{inputenc}
\usepackage[a4paper, top=0.6in, right=0.6in, left=0.4in,bottom= 0.6in]{geometry}
% \usepackage{wrapfig, blindtext}
% \usepackage{graphicx}

%% packages

% \usepackage{blindtext} % needed for creating dummy text passages
%\usepackage{ngerman} % needed for German default language
\usepackage{amsmath} % needed for command eqref
\usepackage{amssymb} % needed for math fonts
\usepackage[
	colorlinks=true
	,breaklinks
	%,ngerman
	]{hyperref} % needed for creating hyperlinks in the document, the option colorlinks=true gets rid of the awful boxes, breaklinks breaks lonkg links (list of figures), and ngerman sets everything for german as default hyperlinks language
% \usepackage[hyphenbreaks]{breakurl} % ben�tigt f�r das Brechen von URLs in Literaturreferenzen, hyphenbreaks auch bei links, die �ber eine Seite gehen (mit hyphenation).
\usepackage{xcolor}
\definecolor{c1}{rgb}{0,0,1} % blue
\definecolor{c2}{rgb}{0,0.3,0.9} % light blue
\definecolor{c3}{rgb}{0.3,0,0.9} % red blue
\hypersetup{
    linkcolor={c1}, % internal links
    citecolor={c2}, % citations
    urlcolor={c3} % external links/urls
}
%\usepackage{cite} % needed for cite
\usepackage[round,authoryear]{natbib} % needed for cite and abbrvnat bibliography style
\usepackage[nottoc]{tocbibind} % needed for displaying bibliography and other in the table of contents
\usepackage{graphicx} % needed for \includegraphics 
\usepackage{longtable} % needed for long tables over pages
\usepackage{bigstrut} % needed for the command \bigstrut
\usepackage{enumerate} % needed for some options in enumerate
\usepackage{todonotes} % needed for todos
\usepackage{makeidx} % needed for creating an index
\usepackage{fancyhdr}
\usepackage{float}

\pagestyle{fancy}
\fancyhf{}
% \fancyfoot[]{Hello} 

\renewcommand{\headrulewidth}{0pt}
\renewcommand{\footrulewidth}{0pt}
\fancyfoot[c]{\thepage}

\title{Classical Mechanics}
\author{Devansh Shukla}
\date{\today}

\usepackage{indentfirst}
\usepackage{sectsty}

\sectionfont{\fontsize{12}{14}\selectfont}
\subsectionfont{\fontsize{10}{12}\selectfont}
\subsubsectionfont{\fontsize{8}{10}\selectfont}

\newcommand{\Lagr}{\mathcal{L}}
\newcommand{\ddt}{\frac{d}{dt}}
\newcommand{\ddtf}[1]{\frac{d #1}{dt}}
\newcommand{\pdt}[2]{\frac{\partial #1}{\partial #2}}
\newcommand{\half}{\frac{1}{2}}

\usepackage{titlesec}
\titleclass{\subsubparagraph}{straight}[\subparagraph]
\newcounter{subsubparagraph}
\renewcommand{\thesubsubparagraph}{\Alph{subsubparagraph}}
\titleformat{\subsubparagraph}[runin]{\normalfont\normalsize\bfseries}{\thesubsubparagraph}{1em}{}
\titlespacing*{\subsubparagraph} {\parindent}{3.25ex plus 1ex minus .2ex}{1em}

\begin{document}

	\maketitle

	\section*{Calculation of the shortest length in a 2D Eucledian Space: }
		\noindent
		
		$$ l = \int \sqrt{dx^2 dy^2} $$
		$$ l = \int \sqrt{1 + y'^2} $$
		$$ {\delta}l = {\delta}\int \sqrt{1 + y'^2} $$
		$$ l \rightarrow l + O(\delta y^2) $$

		% $$ \therefore generally $$
	
	\section*{Energies: }
		\subsection*{Potential Energy}
			Energy associated with the body/particle due to virtue of its position or configuration.
		\subsection*{Kinetic Energy}
			Energy associated with the body/particle due to virtue of its motion.

	\section*{Generalized Coordinates: }
		\noindent

		Generalized coordinates are a set of coordinates which can fully describe the system. If there are no holonomic constraints, then the no. of Generalised coordinates requires is same as the Degrees of Freedom. They need not be Orthogonal coordinates and can be mixed(mixture of coordinate systems is allowed eq. from cartesian and spherical coordinate systems).
		
		\paragraph*{Advantages:}
		\begin{itemize}
			% \item Since the Equation of Motion is in the form of generalised coordinates it can be easily transformed to another coordinate system.
			% \item Since the Lagrangian and Hamiltonian Equations are in the form of generalised coordinates they remanin invarient during a coordinate transformation.
			\item If a Lagrangian is specified in generalised coordinates then its very easy to rewrite it in another coordinate system.
			\item It shows the generality of the equations, I mean, it shows that (for eq)Principle of Stationary Action doesn't depend upon the coordinate system used, be it Cartesian or Curviliear, the equations would correspond to the same thing.
			\item Its easier to handle holonomic constraints with Generalised Coordinates.
		\end{itemize}
		
		\paragraph*{Diff from Ordinary Coordinate System}
		\begin{itemize}
			\item Its difficult to handle constraints in ordinary coordinate systems.
			\item The dimensions of Generalised Coordinates may differ from dimensions of Ordinary Coordinate Systems.
			% \item 
		\end{itemize}

	\section*{Constraints: }
		
		\noindent	

		A Constraint is nothing but a limitation on the range of motion of any particle/body, it can restrict the degrees of freedom of the particle/body.
		\paragraph*{Types of Constraints: }

		\subparagraph*{Holonomic Constraints: }
			Constraints of the form $f(x_i, t) = 0$ are called holonomic constraints, these can be expressed using an equation. Holonomic constrains restricts the degrees of freedom, suppose a system has $3n$ degrees of freedom and $m$ holonomic constraints then the $dof$ would reduce to $3n - m$.
			
			\subsubparagraph*{Examples: }

			\begin{itemize}
				\item Consider a pendulum with a bob and an inextendable, massless string, then there will be two holonomic constraints, (i) The Length of the string $\sqrt{x^2+y^2}$, (ii) The Motion along $y$ would be zero, $\dot{y} = 0$.
				\item Consider a small cube which can slide over a wedge, in this case, there will be $1$ holonomic constraint, the cube cannot have motion along the $y$ direction, $\dot{y}=0$.
				\item In the famous Catenary problem, there will be $1$ holonomic constraint and $2$ boundary conditions, the constraint would be the length of the string.
			\end{itemize}

		\subparagraph*{Non-Holonomic Constraints: }
			Constraints not of the form $f(x_i, t) = 0$ are called non-holonomic constraints, these can be expressed using an inequality or by a differential equation which can only be solved by solveing the problem first. These constraints are difficult to solve and require unique approach to each problem. Also, the no. of degrees of freedom remains same in this case.

			\subsubparagraph*{Examples: }

			\begin{itemize}
				\item A particle confined into a box of length a,b,c , \[\frac{-a}{2}\leq x \leq \frac{a}{2}, \frac{-b}{2}\leq y \leq\frac{b}{2}, \frac{-c}{2}\leq z \leq \frac{c}{2}\]			
				\item A cylinder of radius R is rolling down a wedge with max height h, the constraint in this case will be the rolling condition $\dot{r} = \dot{\theta} R$.
			\end{itemize}
		
		\subparagraph*{Rheonomous Constraints: }
			Time Dependent Constraints are called Rheonomous Constraints.

			\subsubparagraph*{Examples: }

			\begin{itemize}
				\item Gas Expanding/Contracting in a container (Constraint: Volume of the Gas).
				\item A pendulum whose length varies w.r.t time (Constraint: Length of the pendulum).
				\item A particle which should cling on to a curve only until a certain time $t=t_1$.
				% \item 
			\end{itemize}

		\subparagraph*{Scleronomous Constraints: }
			Time Independent Constraints are called Scleronomous Constraints.

			\subparagraph*{Examples: }

			\begin{itemize}
				\item A pendulum with fixed length (Constraint: Length).
				\item A particle confined to the surface of a parabola (Constraint: Equation of Parabola).
			\end{itemize}

	\section*{Virtual Work Prinicple and D'Alembert's Principle: }
		
		\subsection*{Virtual Displacements: }
		Virtual infinitesimal displacements of a system refers to the change in config of the system when there is some arbitary small change in the coordinates, consistant with the forces and constraints imposed on the system at time instant t. This Virtual Displacement is different from actual displacement which takes time $dt$ to complete.
		
		\subsection*{Virtual Work Principle: }
			\textbf{Virtual Work is basically the work due to the virtual displacement, when the time interval $dt=0$. It provides the work done when the system is static(in equilibrium).}
			\textbf{For a system in static equilibrium(Rest), the Virtual Work due to infinitesimal virtual displacement is zero.}

		\subsection*{D'Alembert Principle: } 
			\textbf{The total virtual work of the impressed forces plus the inertial forces vanishes for reversible displacements.}
			\textbf{For a system in dynamic equilibrium the Virtual Work due to all impressed and inertial forces is zero for infinitesimal virtual displacements.}
			
		D'Alembert's Principle provides a more general form of the Hamiltonian Principle with the inclusion of holonomic constraints. It is basically the dynamic version of the static Principle of Virtual Works(since it assumes equilibrium).
		It isn't Applicable to irreversible forces such as sliding friction and is more general than Hamilton's Principle.

		\subsection*{Proof: }

			Consider the system is in equlibrium, then the virual work will be given by the dot product of the total force acting time the virtual displacement,

			$$ W_v = \sum_i \vec{F_i} \cdot \delta \vec{r_i} = 0 $$

			Here, $\vec{F_i}$ is made up of two forces, the applied force $\vec{F_i^a}$ and the force of constraints $\vec{f_i}$,

			Hence, the Virtual Work becomes,

			$$ W_v = \sum_i \vec{F_i^a} \cdot \delta \vec{r_i} + \sum_i \vec{f_i} \cdot \delta \vec{r_i} $$

			We now consider that the net work done due to the forces of constraints is zero, 

			$$ \sum_i \vec{f_i} \delta \vec{r_i} = 0 $$

			Also, since the system is in equilibrium, the virtual work due to the applied forces should be $zero$.

			\begin{equation}
				\sum_i \vec{F_i^a} \cdot \delta \vec{r_i} = 0 \label{eq_pvw} 
			\end{equation} 		
			This is known as the Principle of Virtual Work, 
			
			D'Alembert Principle
			
			From the 2nd Law of Motion, $\vec{F_i} = \dot{\vec{p_i}}$
			$$ \implies \vec{F_i} - \dot{\vec{p_i}} = 0 $$
			
			This works for the D'Alembert's principle which considers reverse forces,

			From the Principle of Virtual Work,
				$$ \sum_i ( \vec{F_i} - \dot{\vec{p_i}} = 0 ) $$
			For, $F_i = F_i^a + f_i$ and considering that the virtual work from the forces of constraints vanishes, 

			$$ \sum_i(\vec{F_i^a} - \dot{\vec{p_i}}) \cdot \delta \vec{r_i} = 0 $$
			
			But since the coordinates $\vec{r_i}$ are not independent of each other and are connected by constraints, we consider a set of independent generalised coordinates.

			$$ q_i = f(r_i) $$

			Therefore, the virtual displacement $\delta \vec{r_i}$ is given by,

			$$ \delta r_i = \sum_j \pdt{r_i}{q_j}\delta q_j $$

			Hence, the Virtual Work Becomes,

			$$ W_v = \sum_{i,j} \vec{F_i} \cdot \pdt{r_i}{q_j}\delta q_j $$

			Let, $Q_j = \sum_i \vec{F_i} \cdot \pdt{r_i}{q_j}$, where $Q_j$ are called Generalised Forces(doesn't necessarily have dimensions of force).

			For other term,

			$$ \sum_i \dot{\vec{p_i}} \cdot \delta \vec{r_j}  = \sum_{i,j} m_i \ddot{\vec{r_i}} \cdot \pdt{r_i}{q_j} \delta q_j $$

			On Manupulations,

			$$ \sum_{i,j} m_i \ddot{\vec{r_i}} \cdot \pdt{r_i}{q_j} \delta q_j = \sum_i[ \ddt (m_i \dot{\vec{r_i}}) - m_i \dot{\vec{r_i}} \cdot \ddt(\pdt{\vec{r_j}}{q_j})] $$
			
			Since, the order of $\ddt$ and $\pdt{}{}$ can be changed,

			$$ \ddt(\pdt{\vec{r_i}}{q_j}) = \pdt{\vec{v}}{q_j} $$

			Therefore,

			$$ \sum_{i,j} m_i \ddot{\vec{r_i}} \cdot \pdt{r_i}{q_j} \delta q_j = \sum_i[ \ddt (m_i \vec{v_i}) - m_i \vec{v_i} \cdot \pdt{\vec{v_i}}{q_j}] $$

			On Manipulations, 

			$$ \sum_j [ \ddt(\pdt{\sum_i \half m_i v_i^2}{\dot{q_j}})  - \pdt{\sum_i \half m_i v_i^2}{q_j} - Q_j]\delta q_j $$

			On identifing $T = \sum_i \half m_i v_i^2$ as Kinetic Energy, the Equation of D'Alembert's Principle becomes,

			\begin{equation}
				\sum[\ddt(\pdt{T}{\dot{q_j}}) - \pdt{T}{q_j} - Q_j]\delta q_j = 0 
			\end{equation}
		
	\section*{Lagrangian: }
		% \noindent 
		From the Principle of Stationary Action, $S = \int_{t_1}^{t_2} L dt $.

		Lagrangian is a function which when integrated along the world-line produces Action.

		EOM's can be obtainted from $\Lagr$ on application of Euler-Lagrange Equation, \hfill \\
		
		It should be noted that EOM's
		\begin{itemize}
			\item Remain Invarient even if any constant is added to the Lagrangian.
			\item Remain Invarient even if any constant is multiplied to the Lagrangian.
			\item Remain Invarient even if any derivate of time is added to the Lagrangian, $\Lagr' = \Lagr + \ddtf{f}$. \todo{Proofs}
	
		\end{itemize}

	\section*{Action Priciple: Euler-Lagrange Equation }
		
		Action is defined by $S = \int_a^b \Lagr(q, \dot{q}, t) dt$ where $\Lagr$ is called the Lagrangian and is defined to be a function of $q$, $\dot{q}$, and $t$. \hfill \\

		Suppose, if we assume Action be a line and if we introduce infintisimal variation in it, then the variation of the Action will only be Zero for extremum positions.

		$$ \delta S = \delta \int \Lagr(q, \dot(q), t) dt $$

		Since, $\delta$ and $\int$ commute, the position of both can be interchanged.

		$$ \delta S = \int \delta \Lagr(q, \dot(q), t) dt $$

		From the first principles,

		$$ \delta S = \int dt \frac{\partial\Lagr}{\partial q}\delta q + \frac{\partial\Lagr}{\partial\dot{q}}\delta \dot{q} + \frac{\partial\Lagr}{\partial t}\delta t) $$

		Assuming, the $\delta t$ is $0$.

		$$ \delta S = \int dt \frac{\partial\Lagr}{\partial q}\delta q + \frac{\partial\Lagr}{\partial\dot{q}}\delta \dot{q} $$

		But, $\delta S = 0$

		$$ \therefore \int dt \frac{\partial\Lagr}{\partial q}\delta q + \frac{\partial\ Lagr}{\partial\dot{q}}\delta \dot{q} = 0 $$

		$$ \int dt \frac{\partial\Lagr}{\partial q}\delta q + \frac{\partial\ Lagr}{\partial\dot{q}}\delta \frac{dq/dt} = 0 $$
		
		Since, $\delta$ and $\frac{d}{dt}$ commute,

		$$ \int dt \frac{\partial\Lagr}{\partial q}\delta q + \frac{\partial\ Lagr}{\partial\dot{q}}\ddt\delta q = 0 $$

		$$Using,\ \frac{\partial\Lagr}{\partial \dot{q}}\frac{d}{dt}\delta q = \ddt(\frac{\partial\Lagr}{\partial\dot{q}}\delta q) - \ddt(\frac{\partial\Lagr}{\partial\dot{q}}) \delta q $$

		$$ \int dt \frac{\partial\Lagr}{\partial q}\delta q + \ddt(\frac{\partial\Lagr}{\partial\dot{q}}\delta q) - \ddt(\frac{\partial\Lagr}{\partial\dot{q}}) \delta q = 0 $$

		For Integral from a to b, $\int_a^b dt \delta q = 0$, then

		$$ \int_a^b dt \frac{\partial\Lagr}{\partial q}\delta q - \ddt(\frac{\partial\Lagr}{\partial\dot{q}}) \delta q = 0 $$
		
		Diff both sides by $\ddt$,

		\begin{equation}
			\frac{\partial\Lagr}{\partial q}\delta q - \ddt(\frac{\partial\Lagr}{\partial\dot{q}}) \delta q = 0 \label{eq_E-L}
		\end{equation}

		Obtained is nothing but the Euler-Lagrange Equation.
		
		Suppose, $\Lagr$ is not a functin of q(Generally, the potential term is zero), 
		$$\implies \frac{\partial\Lagr}{\partial q} = 0 $$
		$$\therefore \ddt\frac{\partial\Lagr}{\partial \dot{q}} = 0 $$
		$$\implies \frac{\partial\Lagr}{\partial \dot{q}} = k\ \ where\ k\ is\ a\ constant$$ 

		Suppose, $\Lagr$ is not a function of $\dot{q}$(Generally, the Kinetic Energy term is Zero), then
		$$ \ddot{q}\pdt{L}{\dot{q}} + \dot{q}\pdt{L}{q} - \ddt(\dot{q}\pdt{L}{\dot{q}}) = 0$$

	\section*{Conditional Variation: }
		
		\noindent

		Suppose $ I = \int_a^b F(y, y', x) dx $ with Constrint $ J = \int_a^b G(y, y', x) dx $, then, \hfill \\

		Consider, $ K = I + \lambda J $
		$$\therefore K = \int_a^b (F(y, y', x) + \lambda G(y, y', x)) dx $$
		Since, constrain (J) is a fixed function, then $\delta J =0 $.
		$$\implies \delta K = \delta I $$

		$$ \Lagr = F(y, y', x) + \lambda G(y, y', x) $$ where $\lambda$ is underdetermined Lagrangian constant.
	
	\section*{Lagrangian of a Free Particle: }

		Lagrangian of a Free Particle constitues only from Kinetic Energy since the particle in a constant potential field thus rendering potential energy to zero. 
		More generally, it can be said that, since Potential energy arises from the configuration/ position of the particle, and in case of free particle, the configuration and postion are irrelevent.

		The space for a free paricle is homogenous and isotropic, homogenous is nothing but isotropic at all positions, thus making the argument of position irrelevent.

		Since, the space is homogenous and istropic then the Lagrangian must also not depend on the actual coordinates and directions.

		Thus, the Lagrangian of a Free Particle must only be a function of $\Lagr = \Lagr(\dot{q}^2)$.

		Therefore, 
			$$ \Lagr = k \dot{q}^2 $$
		
		Here, the RHS term signifies the Kinetic Energy with $k=\half m$.

	\section*{Lagrangian of a Charged Particle Moving in a EM field: }

		\todo{Lagrangian in EM Field}

	\section*{Conservation Laws and Symmetries: }

		\noindent

		\paragraph*{Nother's Theorem}
			For every symmetry, there always exists a conserved quantity.
		
		If a $\Lagr$ does not expcilitly depend on a coordinate $\rightarrow$ Conservation Law.

		\subsection*{Translation Symmetry: }

			\noindent

			Suppose, a particle is at $q_i$ with a velocity of $\dot{q_i}$ at t and has $\Lagr = k\dot{q_i}^2$.
			Consider a coordinate transformation where $q_i^* = q_i + \delta$, where $\delta$ is an infintisimal variation, then
			
			$$ q_i^* = q_i + \delta $$
			$$ \dot{q_i^*} = \dot{q_i} $$
			$$\therefore \delta \Lagr = 0 $$

			By applying the Euler-Lagrange Eq., 

			$$ \pdt{L}{\dot{q_i}} = 2k\dot{q_i} = 0 $$
			$$\implies \dot{P_i} = 0 $$

			$$ P_i = Constant $$

			Thus, Linear Momentum is conserved.

		\subsection*{Rotational Symmetry: }

			\noindent

			Suppose, a particle is at x,y has a $\Lagr = k (\dot{x}^2 + \dot{y}^2) - V(x^2 + y^2)$, where $ k = \half m $	

			Consider a rotation by $\delta$ angle, then

			$$ x' = x\cos{\delta} + y\sin{\delta} $$
			$$ y' = -x\sin{\delta} + y\cos{\delta} $$

			Assuming, $\delta$ to be infintisimal, then

			$$ \sin{\delta} \sim \delta\ and\ \cos{\delta} \sim 1 $$

			$$\therefore x' = x + y\delta\ and\ y' = -x\delta + y $$

			Hence, the variation in $x$ is $\delta x = y \delta$ and variation in $y$ is $\delta y = -x \delta $

			$$ \delta \dot{x} = y \delta\ and\ \delta \dot{y} = - x \delta $$

			Therefore, the variation in $x^2 + y^2$is
			
			\begin{equation}
				\begin{split}	
					\delta (x^2 + y^2) & = 2 x \delta x + 2y \delta y \\
					&= 2 x y \delta - 2 x y \delta  \\
					& = 0
				\end{split}
			\end{equation}
			
			Hence, the variation in $V$ should also be $0$.

			For Variation in the Kinetic Term,

			\begin{equation}
				\begin{split}	
					\delta (\dot{x^2} + \dot{y^2}) & = 2 \dot{x} \delta \dot{x} + 2 \dot{y} \delta \dot{y} \\
					& = 2 \dot{x} \dot{y} \delta - 2 \dot{x} \dot{y} \delta \\
					& = 0
				\end{split}
			\end{equation}
			
			Hence, the variation is Kinetic Term is also Zero.

			Therefore, Variation in $\Lagr$ is Zero.

			Thus, for infinitisimal rotation of the system, the Lagrangian is Invarient.
			
		\subsection*{Generalization: }

			\noindent

			Assuming, $q_i$ and $\dot{q_i}$ as the generalized coordinates.

			Suppose, a variation is brought in $q$,
			$$ \delta q = f_i(q) \delta $$ where $\delta$ is nothing but a small change, and is a constant. \hfill \\
			Consider $\epsilon = \delta$ to reduce the clutter in notation.

			It is to be noted that this infintisimal change can be stacked up to form a finite change, still these eqs will hold up, since, if $V(x, y)$ is the potential and $\delta V = \pdt{V}{x}\delta x = 0$, $\implies \pdt{V}{x}$ is $0$ at all points.

			Now,

			\begin{eqnarray}
				\delta q = f_i(q) \epsilon \label{eq_q} \\
				\delta \dot{q} = \ddt(f_i(q) \epsilon) \label{eq_dotq}
			\end{eqnarray}
			
			Then,

			$$ \delta \Lagr = \pdt{\Lagr}{q_i}\delta q_i + \pdt{\Lagr}{\dot{q_i}}\delta \dot{q_i} + \pdt{\Lagr}{t} \delta t $$
			Assuming, $\Lagr$ is Independent of time. (Assuming time symmetry)

			$$ \delta \Lagr = \pdt{\Lagr}{q_i}\delta q_i + \pdt{\Lagr}{\dot{q_i}}\delta \dot{q_i} $$
			Since, $\pdt{\Lagr}{\dot{q}}$ is nothing but the cannonical conjugate of Momentum and $\pdt{\Lagr}{q_i}=\ddt \pdt{\Lagr}{\dot{q}}$\eqref{eq_E-L}

			\begin{equation}
				\delta \Lagr = \dot{P_i}\delta q_i + P_i \delta \dot{q_i} \label{eq_var_L}
			\end{equation}

			Using, \eqref{eq_q} and \eqref{eq_dotq}

			$$ \delta \Lagr = \dot{P_i} f_i(q) \epsilon + P_i \ddt(f_i(q)\epsilon) $$

			$$ \delta \Lagr = \epsilon \ddt(P_i f_i(q)) $$
			
			Assuming, there is a symmetry, then $\delta \Lagr = 0$,

			$$\implies \epsilon \ddt{P_i f_i(q)} = 0$$

			$$ \ddt{P_i f_i(q)} = 0 $$

			Therefore, the \textbf{conserved quantity} is,

			\begin{equation}
				Q = P_i f_i(q) \label{eq_conserved_Q}
			\end{equation}
		
		\subsection*{Time Symmetry: } \label{time_symmetry}
			\noindent \\

			% When $\Lagr$ doesn't expcilitly depend upon $t$, or $\Lagr = \Lagr(q, \dot{q}))$.

			% $$ \ddt(\Lagr) = \pdt{\Lagr}{q}\dot{q} + \pdt{\Lagr}{\dot{q}} \ddot{q} $$

			% But, since $ P = \pdt{\Lagr}{\dot{q}} $ and $ \dot{P} = \pdt{\Lagr}{q} $,

			% $$ \ddt(\Lagr) = \dot{P} q + P \ddot{q} $$

			% $$ \ddt(\Lagr) = \ddt(P \dot{q}) $$

			% Therefore, the conserved quantity is, 
			% \begin{equation}
			% 	\ddt(\Lagr - P \dot{q}) = 0
			% \end{equation}

			% which is nothing but the total energy, called Hamiltonian.

			% $$ -H = \Lagr - P \dot{q} $$
			% \begin{equation}
			% 	\implies H = P \dot{q} - \Lagr
			% \end{equation}

			When $\Lagr$ doesn't expcilitly depend upon $t$, or $\Lagr = \sum_i \Lagr(q_i, \dot{q_i}))$. ($\sum_i$ is ommited in later text to reduce the clutter)

			$$ \ddtf{\Lagr} = \pdt{\Lagr}{q_i}\dot{q_i} + \pdt{\Lagr}{\dot{q_i}} \ddot{q_i} $$

			But, since $ P_i = \pdt{\Lagr}{\dot{q_i}} $ and $ \dot{P_i} = \pdt{\Lagr}{q_i} $,

			$$ \ddtf{\Lagr} = \dot{P_i} q_i + P_i \ddot{q_i} $$

			$$ \ddtf{\Lagr} = \ddt(P_i \dot{q_i}) $$

			Therefore, the conserved quantity is,

			\begin{equation}
				\ddt(\Lagr - P_i \dot{q_i}) = 0
			\end{equation}

			which is nothing but the total energy, called Hamiltonian.

			$$ -H = \Lagr - P_i \dot{q_i} $$
			\begin{equation}
				\implies H = P_i \dot{q_i} - \Lagr
			\end{equation}


			\subsubsection*{Hamiltonian of a particle moving linearly in a potential field: }
				\noindent \\

				For such a particle, $\Lagr = \half m \dot{x}^2 - V(x)$, where $V(x)$ is the potential energy at position $x$.

				$$ \therefore H = m\dot{x}^2 - \half m \dot{x}^2 + V(x) $$
				$$ \implies H = \half m \dot{x}^2 + V(x) $$

			\subsubsection*{Assuming Time Dependence: }
				\noindent \\

				Suppose, the $\Lagr$ depends upon time.
				% Assume a harmonic oscillator, with spring constant which depends upon time.

				% $$ \Lagr = \half m \dot{x}^2 - \half k(t) x^2 $$

				% $$ \therefore \ddtf{\Lagr} = \pdt{\Lagr}{q}\dot{q} + \pdt{\Lagr}{\dot{q}}\ddot{q} + \pdt{\Lagr}{t} $$
				% $$ \implies \ddtf{\Lagr} = \ddt(P \dot{q}) + \pdt{\Lagr}{t} $$

				% $$ \ddtf{\Lagr - P \dot{q}} = \pdt{\Lagr}{t} $$

				% \begin{equation}
				% 	\ddtf{H} = - \pdt{\Lagr}{t}
				% \end{equation}

				$$ \therefore \ddtf{\Lagr} = \pdt{\Lagr}{q_i}\dot{q_i} + \pdt{\Lagr}{\dot{q_i}}\ddot{q_i} + \pdt{\Lagr}{t} $$
				$$ \implies \ddtf{\Lagr} = \ddt(P_i \dot{q_i}) + \pdt{\Lagr}{t} $$

				$$ \ddtf{\Lagr - P_i \dot{q_i}} = \pdt{\Lagr}{t} $$

				\begin{equation}
					\ddtf{H} = - \pdt{\Lagr}{t}
				\end{equation}


	\paragraph*{Lagrangian in a Rotational Transformation: }\noindent\hfill\\
		 
		Suppose, a particle is at x,y has a $\Lagr = k (\dot{x}^2 + \dot{y}^2)$, where $ k = \half m $.

		Consider a coordinate system, {x', y'}, with the same origin but rotating with $\omega$ angular velocity.

		Therefore, for the transformation eqs. will be,

		\begin{eqnarray}
			x' = x\cos{\omega t} + y\sin{\omega t}\label{eq:rot1} \\
			y' = -x\sin{\omega t} + y\cos{\omega t}\label{eq:rot2}
		\end{eqnarray}
		

		Using \eqref{eq:rot1} and \eqref{eq:rot2},

		\begin{equation}
			\Lagr = \half m (\dot{x}^2 + \dot{y}^2) + \half m \omega^2(x^2 + y^2) + m \omega (\dot{x}y - x\dot{y})
		\end{equation}

		Since, $ \Lagr = T - V $, the first term refers to the original $\Lagr$ which is nothing but the Kinetic Energy, the second terms seems to be similar to Potential Energy and the third term corresponds to Correlios Force.


	\section*{Hamiltonian: }
		\noindent \hfill

		Consider a $\Lagr = \Lagr(q, \dot(q), t)$.

		Then, then $H$ is defined by, 

		\begin{equation}
			H = \sum_i P_i \dot{q_i} - \Lagr
		\end{equation}

		For small variation in $H$,
		\begin{equation}
			\delta H = \pdt{H}{P} \delta P + \pdt{H}{q}\delta q \label{eq_var_H}
		\end{equation}
		$$ \delta H = \sum_i \delta P_i\dot{q_i} - \delta \Lagr $$
		$$ \delta H = \sum_i \delta(P_i) \dot{q_i} + P_i\delta(\dot{q_i}) - \pdt{\Lagr}{q_i}\delta q_i - \pdt{\Lagr}{\dot{q_i}}\delta \dot{q_i} $$

		Using \eqref{eq_var_L}, the $2nd$ and $4th$ turn cancels each other out, 

		$$ \delta H = \sum_i \dot{q_i} \delta(P_i) - \dot{P_i} \delta(q_i) $$

		Comparing this equation, with \eqref{eq_var_H}, ({Dropping $\sum$ notation to increase clarity.})

		\begin{eqnarray}
			\pdt{H}{P} = \dot{q} \\
			\pdt{H}{q} = -\dot{P}
		\end{eqnarray}

		\paragraph*{Note:}
			The no. of Hamiltonian equations is double the Lagrangian equations. \\
			The Hamiltonian eqs are $1st$ Order Differential Equations where as the Lagrangian eqs are $2nd$ Order.
	
	\section*{Integrals of Motion}
		
		For a system with Lagrangian $\Lagr$, there exists function of $q$ and $\dot{q}$ which are conserved called integrals of motion.

		For a system of s degrees of freedom, there exits 2s-1 such functions.

		Example: Energy, Momentum, Angular Momentum, etc.

		The conservations of energy directly arises due to symmetry in time, such systems are called conservative systems. \autoref{time_symmetry}

	\section*{Poisson's Brackets}
		
		Poisson Brackets are Jacobian w.r.t generalized coordinate and its corresponding cannonical conjugate of momentum.

		\[
			(f,g) = 
			\begin{vmatrix}
				\pdt{f}{q} & \pdt{f}{p} \\
				\pdt{g}{q} & \pdt{g}{p}
			\end{vmatrix}
			\]
		
		\subsection*{Some Properties}

			\begin{equation}
				(f, g) = - (g, f)
			\end{equation}

			\begin{equation}
				(f_1 + f_2, g) = (f_1, g) + (f_2, g) \footnote{Comes directly from distributive property of $\partial$}
			\end{equation}

			\begin{equation}
				(f_1 f_2, g) = f_2(f_1, g) + f_1(f_2, g) \footnote{Direct result from commutative proper of $\partial$}	
			\end{equation}
			
			\begin{equation}
				(f, (g, h)) + (h, (f, g)) + (g, (h, f)) = 0	\label{eq_jacobi_identity}
			\end{equation}
			
			\autoref{eq_jacobi_identity} is called the \textbf{Jacobi Identity}.
		

			\begin{eqnarray}
				(f, q) = - \pdt{f}{p} \\
				(f, p) = \pdt{f}{q}
			\end{eqnarray}

			Hence we find,

			\begin{equation}
				(q_{\lambda}, q_{\sigma}) = 0, \hspace*{1cm} (p_{\lambda}, p_{\sigma}) = 0, \hspace*{1cm} (q_{\lambda}, p_{\sigma}) = \delta_{\lambda \sigma}
			\end{equation}

\end{document}
