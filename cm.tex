\documentclass[a4paper]{article}


\usepackage[utf8]{inputenc}
\usepackage[a4paper, top=0.6in, right=0.6in, left=0.4in,bottom= 0.6in]{geometry}
% \usepackage{wrapfig, blindtext}
% \usepackage{graphicx}

%% packages

% \usepackage{blindtext} % needed for creating dummy text passages
%\usepackage{ngerman} % needed for German default language
\usepackage{amsmath} % needed for command eqref
\usepackage{amssymb} % needed for math fonts
\usepackage[
	colorlinks=true
	,breaklinks
	%,ngerman
	]{hyperref} % needed for creating hyperlinks in the document, the option colorlinks=true gets rid of the awful boxes, breaklinks breaks lonkg links (list of figures), and ngerman sets everything for german as default hyperlinks language
% \usepackage[hyphenbreaks]{breakurl} % ben�tigt f�r das Brechen von URLs in Literaturreferenzen, hyphenbreaks auch bei links, die �ber eine Seite gehen (mit hyphenation).
\usepackage{xcolor}
\definecolor{c1}{rgb}{0,0,1} % blue
\definecolor{c2}{rgb}{0,0.3,0.9} % light blue
\definecolor{c3}{rgb}{0.3,0,0.9} % red blue
\hypersetup{
    linkcolor={c1}, % internal links
    citecolor={c2}, % citations
    urlcolor={c3} % external links/urls
}
%\usepackage{cite} % needed for cite
\usepackage[round,authoryear]{natbib} % needed for cite and abbrvnat bibliography style
\usepackage[nottoc]{tocbibind} % needed for displaying bibliography and other in the table of contents
\usepackage{graphicx} % needed for \includegraphics 
\usepackage{longtable} % needed for long tables over pages
\usepackage{bigstrut} % needed for the command \bigstrut
\usepackage{enumerate} % needed for some options in enumerate
\usepackage{todonotes} % needed for todos
\usepackage{makeidx} % needed for creating an index
\usepackage{fancyhdr}
\usepackage{float}

\pagestyle{fancy}
\fancyhf{}
% \fancyfoot[]{Hello} 

\renewcommand{\headrulewidth}{0pt}
\renewcommand{\footrulewidth}{0pt}
\fancyfoot[c]{\thepage}

\title{Classical Mechanics}
\author{Devansh Shukla}
\date{\today}

\usepackage{sectsty}

\sectionfont{\fontsize{12}{14}\selectfont}
\subsectionfont{\fontsize{10}{12}\selectfont}
\subsubsectionfont{\fontsize{8}{10}\selectfont}

\newcommand{\Lagr}{\mathcal{L}}
\newcommand{\ddt}{\frac{d}{dt}}
\newcommand{\ddtf}[1]{\frac{d #1}{dt}}
\newcommand{\pdt}[2]{\frac{\partial #1}{\partial #2}}
\newcommand{\half}{\frac{1}{2}}

% \renewcommand{\paragraph*[1]{\paragraph*{#1}\noindent\\}

\begin{document}

	\maketitle

	\section*{Calculation of the shortest length in a 2D Eucledian Space: }
		\noindent
		
		$$ l = \int \sqrt{dx^2 dy^2} $$
		$$ l = \int \sqrt{1 + y'^2} $$
		$$ {\delta}l = {\delta}\int \sqrt{1 + y'^2} $$
		$$ l \rightarrow l + O(\delta y^2) $$

		% $$ \therefore generally $$
	
	\section*{Energies: }
		\subsection*{Potential Energy}
			Energy associated with the body/particle due to virtue of its position or configuration.
		\subsection*{Kinetic Energy}
			Energy associated with the body/particle due to virtue of its motion.

	\section*{Generalized Coordinates: }

		\noindent
		Independent Coordinates

	\section*{Action Priciple: Euler-Lagrange Equation }
		
		\noindent
		
		Action is defined by $S = \int_a^b \Lagr(q, \dot{q}, t) dt$ where $\Lagr$ is called the Lagrangian and is defined to be a function of $q$, $\dot{q}$, and $t$. \hfill \\

		Suppose, if we assume Action be a line and if we introduce infintisimal variation in it, then the variation of the Action will only be Zero for extremum positions.

		$$ \delta S = \delta \int \Lagr(q, \dot(q), t) dt $$

		Since, $\delta$ and $\int$ commute, the position of both can be interchanged.

		$$ \delta S = \int \delta \Lagr(q, \dot(q), t) dt $$

		From the first principles,

		$$ \delta S = \int dt \frac{\partial\Lagr}{\partial q}\delta q + \frac{\partial\Lagr}{\partial\dot{q}}\delta \dot{q} + \frac{\partial\Lagr}{\partial t}\delta t) $$

		Assuming, the $\delta t$ is $0$.

		$$ \delta S = \int dt \frac{\partial\Lagr}{\partial q}\delta q + \frac{\partial\Lagr}{\partial\dot{q}}\delta \dot{q} $$

		But, $\delta S = 0$

		$$ \therefore \int dt \frac{\partial\Lagr}{\partial q}\delta q + \frac{\partial\ Lagr}{\partial\dot{q}}\delta \dot{q} = 0 $$

		$$ \int dt \frac{\partial\Lagr}{\partial q}\delta q + \frac{\partial\ Lagr}{\partial\dot{q}}\delta \frac{dq/dt} = 0 $$
		
		Since, $\delta$ and $\frac{d}{dt}$ commute,

		$$ \int dt \frac{\partial\Lagr}{\partial q}\delta q + \frac{\partial\ Lagr}{\partial\dot{q}}\ddt\delta q = 0 $$

		$$Using,\ \frac{\partial\Lagr}{\partial \dot{q}}\frac{d}{dt}\delta q = \ddt(\frac{\partial\Lagr}{\partial\dot{q}}\delta q) - \ddt(\frac{\partial\Lagr}{\partial\dot{q}}) \delta q $$

		$$ \int dt \frac{\partial\Lagr}{\partial q}\delta q + \ddt(\frac{\partial\Lagr}{\partial\dot{q}}\delta q) - \ddt(\frac{\partial\Lagr}{\partial\dot{q}}) \delta q = 0 $$

		For Integral from a to b, $\int_a^b dt \delta q = 0$, then

		$$ \int_a^b dt \frac{\partial\Lagr}{\partial q}\delta q - \ddt(\frac{\partial\Lagr}{\partial\dot{q}}) \delta q = 0 $$
		
		Diff both sides by $\ddt$,

		\begin{equation}
			\frac{\partial\Lagr}{\partial q}\delta q - \ddt(\frac{\partial\Lagr}{\partial\dot{q}}) \delta q = 0 \label{eq_E-L}
		\end{equation}

		Obtained is nothing but the Euler-Lagrange Equation.
		
		Suppose, $\Lagr$ is not a functin of q(Generally, the potential term is zero), 
		$$\implies \frac{\partial\Lagr}{\partial q} = 0 $$
		$$\therefore \ddt\frac{\partial\Lagr}{\partial \dot{q}} = 0 $$
		$$\implies \frac{\partial\Lagr}{\partial \dot{q}} = k\ \ where\ k\ is\ a\ constant$$ 

		Suppose, $\Lagr$ is not a function of $\dot{q}$(Generally, the Kinetic Energy term is Zero), then
		$$ \ddot{q}\pdt{L}{\dot{q}} + \dot{q}\pdt{L}{q} - \ddt(\dot{q}\pdt{L}{\dot{q}}) = 0$$

	\section*{Conditional Variation: }
		
		\noindent

		Suppose $ I = \int_a^b F(y, y', x) dx $ with Constrint $ J = \int_a^b G(y, y', x) dx $, then, \hfill \\

		Consider, $ K = I + \lambda J $
		$$\therefore K = \int_a^b (F(y, y', x) + \lambda G(y, y', x)) dx $$
		Since, constrain (J) is a fixed function, then $\delta J =0 $.
		$$\implies \delta K = \delta I $$

		$$ \Lagr = F(y, y', x) + \lambda G(y, y', x) $$ where $\lambda$ is underdetermined Lagrangian constant.

	\section*{Conservation Laws and Symmetries: }

		\noindent

		\paragraph*{Nother's Theorem}
			For every symmetry, there always exists a conserved quantity.
		
		If a $\Lagr$ does not expcilitly depend on a coordinate $\rightarrow$ Conservation Law.

		\subsection*{Translation Symmetry: }

			\noindent

			Suppose, a particle is at $q_i$ with a velocity of $\dot{q_i}$ at t and has $\Lagr = k\dot{q_i}^2$.
			Consider a coordinate transformation where $q_i^* = q_i + \delta$, where $\delta$ is an infintisimal variation, then
			
			$$ q_i^* = q_i + \delta $$
			$$ \dot{q_i^*} = \dot{q_i} $$
			$$\therefore \delta \Lagr = 0 $$

			By applying the Euler-Lagrange Eq., 

			$$ \pdt{L}{\dot{q_i}} = 2k\dot{q_i} = 0 $$
			$$\implies \dot{P_i} = 0 $$

			$$ P_i = Constant $$

			Thus, Linear Momentum is conserved.

		\subsection*{Rotational Symmetry: }

			\noindent

			Suppose, a particle is at x,y has a $\Lagr = k (\dot{x}^2 + \dot{y}^2) - V(x^2 + y^2)$, where $ k = \half m $ \todo{Check With Potential too.}

			Consider a rotation by $\delta$ angle, then

			$$ x' = x\cos{\delta} + y\sin{\delta} $$
			$$ y' = -x\sin{\delta} + y\cos{\delta} $$

			Assuming, $\delta$ to be infintisimal, then

			$$ \sin{\delta} \sim \delta\ and\ \cos{\delta} \sim 1 $$

			$$\therefore x' = x + y\delta\ and\ y' = -x\delta + y $$

			Hence, the variation in $x$ is $\delta x = y \delta$ and variation in $y$ is $\delta y = -x \delta $

			$$ \delta \dot{x} = y \delta\ and\ \delta \dot{y} = - x \delta $$

			Therefore, the variation in $x^2 + y^2$is
			
			\begin{equation}
				\begin{split}	
					\delta (x^2 + y^2) & = 2 x \delta x + 2y \delta y \\
					&= 2 x y \delta - 2 x y \delta  \\
					& = 0
				\end{split}
			\end{equation}
			
			Hence, the variation in $V$ should also be $0$.

			For Variation in the Kinetic Term,

			\begin{equation}
				\begin{split}	
					\delta (\dot{x^2} + \dot{y^2}) & = 2 \dot{x} \delta \dot{x} + 2 \dot{y} \delta \dot{y} \\
					& = 2 \dot{x} \dot{y} \delta - 2 \dot{x} \dot{y} \delta \\
					& = 0
				\end{split}
			\end{equation}
			
			Hence, the variation is Kinetic Term is also Zero.

			Therefore, Variation in $\Lagr$ is Zero.

			Thus, for infinitisimal rotation of the system, the Lagrangian is Invarient.
			
		\subsection*{Generalization: }

			\noindent

			Assuming, $q_i$ and $\dot{q_i}$ as the generalized coordinates.

			Suppose, a variation is brought in $q$,
			$$ \delta q = f_i(q) \delta $$ where $\delta$ is nothing but a small change, and is a constant. \hfill \\
			Consider $\epsilon = \delta$ to reduce the clutter in notation.

			It is to be noted that this infintisimal change can be stacked up to form a finite change, still these eqs will hold up, since, if $V(x, y)$ is the potential and $\delta V = \pdt{V}{x}\delta x = 0$, $\implies \pdt{V}{x}$ is $0$ at all points.

			Now,

			\begin{eqnarray}
				\delta q = f_i(q) \epsilon \label{eq_q} \\
				\delta \dot{q} = \ddt(f_i(q) \epsilon) \label{eq_dotq}
			\end{eqnarray}
			
			Then,

			$$ \delta \Lagr = \pdt{\Lagr}{q_i}\delta q_i + \pdt{\Lagr}{\dot{q_i}}\delta \dot{q_i} + \pdt{\Lagr}{t} \delta t $$
			Assuming, $\Lagr$ is Independent of time. (Assuming time symmetry)

			$$ \delta \Lagr = \pdt{\Lagr}{q_i}\delta q_i + \pdt{\Lagr}{\dot{q_i}}\delta \dot{q_i} $$
			Since, $\pdt{\Lagr}{\dot{q}}$ is nothing but the cannonical conjugate of Momentum and $\pdt{\Lagr}{q_i}=\ddt \pdt{\Lagr}{\dot{q}}$\eqref{eq_E-L}

			\begin{equation}
				\delta \Lagr = \dot{P_i}\delta q_i + P_i \delta \dot{q_i} \label{eq_var_L}
			\end{equation}

			Using, \eqref{eq_q} and \eqref{eq_dotq}

			$$ \delta \Lagr = \dot{P_i} f_i(q) \epsilon + P_i \ddt(f_i(q)\epsilon) $$

			$$ \delta \Lagr = \epsilon \ddt(P_i f_i(q)) $$
			
			Assuming, there is a symmetry, then $\delta \Lagr = 0$,

			$$\implies \epsilon \ddt{P_i f_i(q)} = 0$$

			$$ \ddt{P_i f_i(q)} = 0 $$

			Therefore, the \textbf{conserved quantity} is,

			\begin{equation}
				Q = P_i f_i(q) \label{eq_conserved_Q}
			\end{equation}
		
		\subsection*{Time Symmetry: }
			\noindent \\

			% When $\Lagr$ doesn't expcilitly depend upon $t$, or $\Lagr = \Lagr(q, \dot{q}))$.

			% $$ \ddt(\Lagr) = \pdt{\Lagr}{q}\dot{q} + \pdt{\Lagr}{\dot{q}} \ddot{q} $$

			% But, since $ P = \pdt{\Lagr}{\dot{q}} $ and $ \dot{P} = \pdt{\Lagr}{q} $,

			% $$ \ddt(\Lagr) = \dot{P} q + P \ddot{q} $$

			% $$ \ddt(\Lagr) = \ddt(P \dot{q}) $$

			% Therefore, the conserved quantity is, 
			% \begin{equation}
			% 	\ddt(\Lagr - P \dot{q}) = 0
			% \end{equation}

			% which is nothing but the total energy, called Hamiltonian.

			% $$ -H = \Lagr - P \dot{q} $$
			% \begin{equation}
			% 	\implies H = P \dot{q} - \Lagr
			% \end{equation}

			When $\Lagr$ doesn't expcilitly depend upon $t$, or $\Lagr = \sum_i \Lagr(q_i, \dot{q_i}))$. ($\sum_i$ is ommited in later text to reduce the clutter)

			$$ \ddtf{\Lagr} = \pdt{\Lagr}{q_i}\dot{q_i} + \pdt{\Lagr}{\dot{q_i}} \ddot{q_i} $$

			But, since $ P_i = \pdt{\Lagr}{\dot{q_i}} $ and $ \dot{P_i} = \pdt{\Lagr}{q_i} $,

			$$ \ddtf{\Lagr} = \dot{P_i} q_i + P_i \ddot{q_i} $$

			$$ \ddtf{\Lagr} = \ddt(P_i \dot{q_i}) $$

			Therefore, the conserved quantity is,

			\begin{equation}
				\ddt(\Lagr - P_i \dot{q_i}) = 0
			\end{equation}

			which is nothing but the total energy, called Hamiltonian.

			$$ -H = \Lagr - P_i \dot{q_i} $$
			\begin{equation}
				\implies H = P_i \dot{q_i} - \Lagr
			\end{equation}


			\subsubsection*{Hamiltonian of a particle moving linearly in a potential field: }
				\noindent \\

				For such a particle, $\Lagr = \half m \dot{x}^2 - V(x)$, where $V(x)$ is the potential energy at position $x$.

				$$ \therefore H = m\dot{x}^2 - \half m \dot{x}^2 + V(x) $$
				$$ \implies H = \half m \dot{x}^2 + V(x) $$

			\subsubsection*{Assuming Time Dependence: }
				\noindent \\

				Suppose, the $\Lagr$ depends upon time.
				% Assume a harmonic oscillator, with spring constant which depends upon time.

				% $$ \Lagr = \half m \dot{x}^2 - \half k(t) x^2 $$

				% $$ \therefore \ddtf{\Lagr} = \pdt{\Lagr}{q}\dot{q} + \pdt{\Lagr}{\dot{q}}\ddot{q} + \pdt{\Lagr}{t} $$
				% $$ \implies \ddtf{\Lagr} = \ddt(P \dot{q}) + \pdt{\Lagr}{t} $$

				% $$ \ddtf{\Lagr - P \dot{q}} = \pdt{\Lagr}{t} $$

				% \begin{equation}
				% 	\ddtf{H} = - \pdt{\Lagr}{t}
				% \end{equation}

				$$ \therefore \ddtf{\Lagr} = \pdt{\Lagr}{q_i}\dot{q_i} + \pdt{\Lagr}{\dot{q_i}}\ddot{q_i} + \pdt{\Lagr}{t} $$
				$$ \implies \ddtf{\Lagr} = \ddt(P_i \dot{q_i}) + \pdt{\Lagr}{t} $$

				$$ \ddtf{\Lagr - P_i \dot{q_i}} = \pdt{\Lagr}{t} $$

				\begin{equation}
					\ddtf{H} = - \pdt{\Lagr}{t}
				\end{equation}


	\paragraph*{Lagrangian in a Rotational Transformation: }\noindent\hfill\\
		 
		Suppose, a particle is at x,y has a $\Lagr = k (\dot{x}^2 + \dot{y}^2)$, where $ k = \half m $.

		Consider a coordinate system, {x', y'}, with the same origin but rotating with $\omega$ angular velocity.

		Therefore, for the transformation eqs. will be,

		\begin{eqnarray}
			x' = x\cos{\omega t} + y\sin{\omega t}\label{eq:rot1} \\
			y' = -x\sin{\omega t} + y\cos{\omega t}\label{eq:rot2}
		\end{eqnarray}
		

		Using \eqref{eq:rot1} and \eqref{eq:rot2},

		\begin{equation}
			\Lagr = \half m (\dot{x}^2 + \dot{y}^2) + \half m \omega^2(x^2 + y^2) + m \omega (\dot{x}y - x\dot{y})
		\end{equation}

		Since, $ \Lagr = T - V $, the first term refers to the original $\Lagr$ which is nothing but the Kinetic Energy, the second terms seems to be similar to Potential Energy and the third term corresponds to Correlios Force.


	\section*{Hamiltonian: }
		\noindent \hfill

		Consider a $\Lagr = \Lagr(q, \dot(q), t)$.

		Then, then $H$ is defined by, 

		\begin{equation}
			H = \sum_i P_i \dot{q_i} - \Lagr
		\end{equation}

		For small variation in $H$,
		\begin{equation}
			\delta H = \pdt{H}{P} \delta P + \pdt{H}{q}\delta q \label{eq_var_H}
		\end{equation}
		$$ \delta H = \sum_i \delta P_i\dot{q_i} - \delta \Lagr $$
		$$ \delta H = \sum_i \delta(P_i) \dot{q_i} + P_i\delta(\dot{q_i}) - \pdt{\Lagr}{q_i}\delta q_i - \pdt{\Lagr}{\dot{q_i}}\delta \dot{q_i} $$

		Using \eqref{eq_var_L}, the $2nd$ and $4th$ turn cancels each other out, 

		$$ \delta H = \sum_i \dot{q_i} \delta(P_i) - \dot{P_i} \delta(q_i) $$

		Comparing this equation, with \eqref{eq_var_H}, ({Dropping $\sum$ notation to increase clarity.})

		\begin{eqnarray}
			\pdt{H}{P} = \dot{q} \\
			\pdt{H}{q} = -\dot{P}
		\end{eqnarray}

		\paragraph*{Note:}
			The no. of Hamiltonian equations is double the Lagrangian equations. \\
			The Hamiltonian eqs are $1st$ Order Differential Equations where as the Lagrangian eqs are $2nd$ Order.




\end{document}
